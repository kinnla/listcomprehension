% Klassenarbeit Mathe 9b 1.12.2016
\documentclass [a4paper,ngerman,12pt]{exam}
\usepackage{amsfonts}
\usepackage{amsmath}
\usepackage {ngerman}
\usepackage [applemac]{inputenc}
\usepackage{tabularx}

\pointpoints{Punkt}{Punkte}
\bonuspointpoints{Bonuspunkt}{Bonuspunkte}
\renewcommand{\solutiontitle}{\noindent\textbf{L�sung:}%
\enspace}

\runningfooter{Klasse 9b}{Klassenarbeit Mathematik, 1.12.2016}{Seite \thepage\ von \numpages}
\runningfootrule

\chqword{Frage}
\chpgword{Seite}
\chpword{Punkte}
\chbpword{Bonus Punkte}
\chsword{Erreicht}
\chtword{Gesamt}

\hpword{Punkte:} % Punktetabelle
\hsword{Ergebnis:}
\hqword{Aufgabe:}
\htword{Summe}

\begin{document}
\noindent {\bf Name}:\\
\noindent {\bf Vorname}:\\
\hrule

\begin{center}
{\large\bf Klassenarbeit zu Potenzen (und Wurzeln)}\\[2mm]
\end{center}
Gymnasium Tiergarten\hfill 1. Dezember 2016\\
Klasse 9b, Mathematik\hfill Bearbeitungszeit: 45 Minuten

\begin{center}
\addpoints\gradetable[h][questions] 
\end{center}
\hrule
\medskip

\begin{questions}
\question[6] Schreibe als Produkt.\\
\begin{parts}
\part $\left(\frac{\sqrt{a}}{b}\right)^3 =$\\[1mm]
\part $-x^2 = $\\[1mm]
\part $(a+b)^{-3} = $\\[1mm]
\end{parts}

\question[6] Schreibe als Potenz.\\
\begin{parts}
\part $a \cdot a \cdot a \cdot a \cdot a \cdot a \cdot a =$\\[1mm]
\part $\sqrt{x-y} = $\\[1mm]
\part $\frac{1}{(a+1) \cdot (a+1) \cdot (a+1) \cdot (a+1)} = $\\[1mm]
\end{parts}

\question[10] Wende die Potenzgesetze an und vereinfache so weit wie m�glich.\\
\begin{parts}
\part $a^5 \cdot a^3 =$\\[1mm]
\part $a^3 : (a\cdot b)^3 =$\\[1mm]
\part ${\left(a^4\right)}^2 =$\\[1mm]
\part $3^{5x} : 3^{2x} =$\\[1mm]
\part $5^a \cdot 3^a =$\\[1mm]
\end{parts}

\pagebreak

\question[4] Schreibe in Normaldarstellung.\\
\begin{parts}
\part Lichtgeschwindigkeit: 300.000.000 m/s =\\
\part Anzahl Galaxien im Universum: 170 Milliarden =\\
\end{parts} 

\question[6] Berechne die Ausdr�cke und wende die Wurzelgesetze an. Schreibe das Ergebnis als ganze Zahl oder Dezimalbruch.\\
\begin{parts}
\part $\sqrt{2} \cdot \sqrt{\frac{9}{2}} =$\\
\part $\sqrt{\left(-3\right)^2} + 123^0 =$\\
\part $\sqrt{7} : \sqrt{28} =$\\
\end{parts}

\question[2] F�r welche Zahl $a \in \mathbb{R}$ gilt $\sqrt{a} = \sqrt{-a}$ ? Begr�nde Deine Aussage.
\vfill

\question[7] Vervollst�ndige die Potenzreihen.

\begin{parts}
\part\Large
\begin{tabular}{ r | c | c | c || c | c | c | c | }
    \hline
    Potenz   & ~~~~~~~ & ~~~~~~~ & \small$\left(\frac{1}{2}\right)^{-1}$\Large & ~~~~~~~ & ~~~~~~~ & ~~~~~~~ & ~~~~~~~ \\ \hline
    Wert & ~~~~~~~ & ~~~~~~~ & ~~~~~~~ & ~~~~~~~ & ~~~~~~~ & \small$\frac{1}{4}$\Large & ~~~~~~~ \\ \hline
  \end{tabular}
  \normalsize
\part\Large
\begin{tabular}{ r | c | c | c || c | c | c | c | }
    \hline
    Potenz   & ~~~~~~~ & ~~~~~~~ & ~~~~~~~ & ~~~~~~~ & ~~~~~~~ & ~~~~~~~ & ~~~~~~~ \\ \hline
    Wert & ~~~~~~~ & ~~~~~~~ & ~~~~~~~ & 1 & -1 & ~~~~~~~ & ~~~~~~~ \\ \hline
  \end{tabular}
\normalsize
\end{parts}

\question[4] Es war einmal ein kluger H�fling, der seinem K�nig ein kostbares Schachbrett schenkte. Der K�nig war �ber den Zeitvertreib sehr dankbar, weil er sich oft ein wenig langweilte. So sprach er zu seinem H�fling: ``Sage mir, wie ich dich f�r dieses wundersch�ne Geschenk belohnen kann.'' Nachdem er eine Weile nachgedacht hatte, sagte der H�fling: ``Nichts weiter will ich, edler Gebieter, als da� Ihr das Schachbrett mit Reis auff�llen m�get. Legt ein Reiskorn auf das erste Feld, und dann auf jedes weitere Feld stets die doppelte Anzahl an K�rnern. Also zwei Reisk�rner auf das zweite Feld, vier Reisk�rner auf das dritte, acht auf das vierte und so fort, bis zum zehnten Feld.'' Dem geschah so.
\begin{parts}
\part Wie viele Reisk�rner befinden sich auf dem zehnten Feld?
\vspace{1cm}
\part Ab welchem Feld befinden sich insgesamt mehr als 100 Reisk�rner auf dem Schachbrett?
\vspace{0.2cm}
\end{parts}

\end{questions}
\end{document}
