% !TEX TS-program = xelatex
% !TEX encoding = UTF-8 Unicode
\documentclass [ngerman,11pt]{article}
\usepackage[a4paper, top=1cm, bottom=1cm, left=2cm, right=1cm]{geometry}
\usepackage[ngerman]{babel}
\usepackage{fontspec}
\usepackage{enumitem}
\usepackage{xcolor}
\usepackage[many]{tcolorbox}
\usepackage{fillwith}
\fillwithset{style=rule}
\usepackage{siunitx}
\sisetup{output-decimal-marker = {,}}

\definecolor{shadecolor}{gray}{.85}
\pagestyle{empty}

%%%%%%%%%%%%%%%%%%%%%%%%%%%%% command definitions
\newcommand{\card}[2]{
\tcboxfit[width=18cm,height=8cm,nobeforeafter,before=\noindent,colback=white]{ 
\begin{minipage}[t][7.3cm][t]{0.482\linewidth}
{\bfseries #1}\\ #2  
\end{minipage}\hspace{3mm}\vline 
}}

\let\oldenumerate\enumerate
\renewcommand{\enumerate}{
\oldenumerate[wide, labelwidth=!, labelindent=0pt]
}


%%%%%%%%%%%%%%%%%%%%%%%%%%%%% document
\begin{document}

\begin{center}
{\bfseries\sffamily \Large\bf Lernkarten A-FRAME}\\[2mm]
von Till Zoppke

\end{center}
\hrule
\medskip

Die Lernkarten wurden erstellt für den Ferienkurs ``ProInformatik V'', 6.-10. August 2018, an der Freien Universität Berlin.

Die Karten gliedern sich in verschiedene Kapitel.
\pagebreak

%%%%%%%%%%%%%%%%%%%%%%%%%%%%% cards

\card{Werkzeuge: Firefox Browser}{
Wir entwickeln ein Internet-Projekt. Oder genauer: eine Website im World-Wide-Web, auf der VR-Inhalte dargestellt werden. Seit der Erfindung des World-Wide-Web hat sich der Web-Browser als Programm zur Darstellung von Websites herausgebildet. Er kann den Seitenquelltext interpretieren und die Seite für den Nutzer darstellen. Für unser Projekt nutzen wir den Webbrowser Firefox. 
\begin{enumerate}
\item Recherchieren Sie im Internet, welche Eigenschaften der Firefox gegenüber anderen Browsern hat, die für unser Projekt vorteilhaft sein könnten.
\item Argumentieren Sie, welchen Sinn es macht, dass wir uns im Projekt auf einen Standard-Browser einigen.
\end{enumerate}
}

\vfill

\tcboxfit[width=18cm,height=8cm,nobeforeafter,
  before=\noindent]{
}

\end{document}
