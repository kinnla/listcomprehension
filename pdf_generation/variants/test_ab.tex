% test doc for variants_ab.py
\documentclass [ngerman,12pt]{paper}
\usepackage[a4paper, total={16cm, 24cm}]{geometry}
\usepackage{amsfonts}
\usepackage{ngerman}
\usepackage{gensymb}
\usepackage[utf8]{inputenc}
\pagestyle{empty}

\begin{document}
\begin{center}
{\large\bf Test Doc for variants\_ab.py}\\[2mm]

\end{center}
\hrule
\medskip

%%%%%%%%%%%%%%%%%%%%%%%%%%%%%%%%%%%%
\begin{enumerate}

\item In variant a, dog bites bird. In variant b, it's the opposite: \texttt{The lazy *A*dog***bird*B* bites the *A*bird***dog*B*}
\item Variant A can be empty: \texttt{*A****This text is shown in Variant B only*B*}
\item Variant B can be empty as well: \texttt{*A*This text is shown in Variant A only****B*}
\item Variants can include multiple lines:\\
\texttt{*A*I\\
write\\
these\\
words\\
in\\
variant A
***
and\\
continue\\
the\\
sentence\\
in\\
variant B
*B*}
\end{enumerate}
\end{document}
